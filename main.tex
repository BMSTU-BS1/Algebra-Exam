\documentclass[a4papper]{article}
\usepackage[left=0.5cm,right=0.5cm,top=2cm,bottom=0.5cm,bindingoffset=0cm]{geometry}
\usepackage{header-algebra}

\title{\Huge Алгебра, Экзамен}

\author{
    Написано простым \href{https://t.me/Borislav_Timoshin}{BIT}-ом \\
    \href{https://github.com/BMSTU-BS1/Linear-Algebra-Exam}{GitHub}.
}

\date{} % Очищаем стандартную дату 

\begin{document}
    \pagestyle{fancy}
    \fancyhead[L]{\thepage}
    \fancyhead[R]{\hyperlink{toc}{\large Содержание}}

    \maketitle

    \epigraph{
        ``Верю, что проникну в пространство, невидимое доселе, и буду творить в нем чудеса!!''.
    }{\rightline{{\rm --- BIT}}}


    \vfill % Добавляем вертикальное заполнение, чтобы сдвинуть вниз

    \begin{center}
        \Large МГТУ им. Н.Э. Баумана, кафедра ФН1. \\
        \emph{2025 — 2026}
    \end{center}

    \newpage
        
    \tableofcontents

    \section{
    Перестановки. Их свойства. Четная и нечетная перестановка. Транспозиция. 
}

% Перестановки. Их свойства.
\subsection{
    Алгебры. Определения и примеры.
}

% Четная и нечетная перестановка.
\subsection{
    Четная и нечетная перестановка.
}

% Транспозиция.
\subsection{
    Теорема Фробениуса (б/д).
}


    \section{
    Разложение перестановки в произведение циклов. Представление перестановки в виде произведения транспозиций. Представление четной перестановки в виде произведения циклов длины 3. 
}

% Разложение перестановки в произведение циклов.
\subsection{
    Разложение перестановки в произведение циклов.
}

% Представление перестановки в виде произведения транспозиций.
\subsection{
    Представление перестановки в виде произведения транспозиций.
}

% Представление четной перестановки в виде произведения циклов длины 3.
\subsection{
    Представление четной перестановки в виде произведения циклов длины 3.
}


    \section{
    Группы. Их свойства. Примеры. Подгруппы. 
}

% Группы. Их свойства. Примеры. Подгруппы.
\subsection{
    Алгебры. Определения и примеры.
}


    \section{
    Действия групп на множествах. Орбиты и стабилизаторы. Их свойства. Формула Бернсайда.
}

% Действия групп на множествах
\subsection{
    Действия групп на множествах.
}

% Орбиты и стабилизаторы. Их свойства
\subsection{
    Орбиты и стабилизаторы. Их свойства.
}
% Формула Бернсайда
\subsection{
    Теорема Фробениуса (б/д).
}


    \section{
    Нормальные подгруппы. Гомоморфизмы групп. Ядро и факторгруппа. Первая теорема о гомоморфизме. 
}

% Нормальные подгруппы.
\subsection{
    Алгебры. Определения и примеры.
}

% Гомоморфизмы групп.
\subsection{
    Теорема Фробениуса (коммутативный случай).
}

% Ядро и факторгруппа.
\subsection{
    Теорема Фробениуса (б/д).
}

% Первая теорема о гомоморфизме.
\subsection{
    Первая теорема о гомоморфизме.
}


    \section{
    Вторая и третья теоремы о гомоморфизме. Теорема Кели.
}

% Вторая и третья теоремы о гомоморфизме.
\subsection{
    Вторая и третья теоремы о гомоморфизме.
}

% Теорема Кели.
\subsection{
    Теорема Фробениуса (коммутативный случай).
}


    \section{
    Конечные абелевы группы. Их классификация.
}

% Конечные абелевы группы. Их классификация.
\subsection{
    Алгебры. Определения и примеры.
}


    \section{
    Свободные абелевы группы. Их базис. Классификация конечно порождённых абелевых групп.
}

% Свободные абелевы группы. Их базис
\subsection{
    Свободные абелевы группы. Их базис.
}

% Классификация конечно порождённых абелевых групп
\subsection{
    Классификация конечно порождённых абелевых групп.
}


    \section{
    Свободные группы. Задание группы образующими и соотношениями. Примеры.
}

% Свободные группы
\subsection{
    Алгебры. Определения и примеры.
}

% Задание группы образующими и соотношениями. Примеры.
\subsection{
    Задание группы образующими и соотношениями. Примеры.
}

    \section{
    Кольца. Определение и основные свойства. Примеры.
}

% Кольца. Определение и основные свойства. Примеры
\subsection{
    Алгебры. Определения и примеры.
}


    \section{
    Идеалы. Факторкольца. Гомоморфизмы колец. Теорема о гомоморфизме для колец. Прямое произведение колец. Группа единиц.
}

% Идеалы
\subsection{
    Алгебры. Определения и примеры.
}

% Факторкольца
\subsection{
    Теорема Фробениуса (коммутативный случай).
}

% Гомоморфизмы колец
\subsection{
    Теорема Фробениуса (б/д).
}

% Теорема о гомоморфизме для колец
\subsection{
    Теорема о гомоморфизме для колец.
}

% Прямое произведение колец
\input{questions/question11/sub-questions/05}
% Группа единиц
\input{questions/question11/sub-questions/06}

    \section{
    Коммутативные кольца. Максимальные и простые идеалы. Их свойства. Критерий того, что факторкольцо является полем.
}

% Коммутативные кольца
\subsection{
    Алгебры. Определения и примеры.
}

% Максимальные и простые идеалы. Их свойства
\subsection{
    Максимальные и простые идеалы. Их свойства.
}
% Критерий того, что факторкольцо является полем
\subsection{
    Критерий того, что факторкольцо является полем.
}

    \section{
    Китайская теорема об остатках. Ее следствия. 
}

% Китайская теорема об остатках. Ее следствия
\subsection{
    Алгебры. Определения и примеры.
}


    \section{
    Главные идеалы. Кольцо главных идеалов. Примеры. Целостные и факториальные кольца. 
}

% Главные идеалы
\subsection{
    Алгебры. Определения и примеры.
}

% Кольцо главных идеалов. Примеры
\subsection{
    Кольцо главных идеалов. Примеры.
}

% Целостные и факториальные кольца
\subsection{
    Целостные и факториальные кольца.
}


    \section{
    НОД. Теорема о том, что любое кольцо главных идеалов факториально. Примеры.
}

% НОД. Теорема о том, что любое кольцо главных идеалов факториально. Примеры.
\subsection{
    Алгебры. Определения и примеры.
}


    \section{
    Локализация. Ее свойства. Примеры.
}

% Локализация. Ее свойства. Примеры
\subsection{
    Алгебры. Определения и примеры.
}


    \section{
    Многочлены. Определения, свойства. Трансцендентные и алгебраические элементы.
}

% Многочлены. Определения, свойства
\subsection{
    Алгебры. Определения и примеры.
}

% Трансцендентные и алгебраические элементы
\subsection{
    Трансцендентные и алгебраические элементы.
}


    \section{
    Алгоритм Евклида. Евклидовы кольца.
}

% Алгоритм Евклида
\subsection{
    Алгебры. Определения и примеры.
}

% Евклидовы кольца
\subsection{
    Теорема Фробениуса (коммутативный случай).
}


    \section{
    Теорема о том, что любое евклидово кольцо является кольцом главных идеалов. Лемма Гаусса.
}

% Теорема о том, что любое евклидово кольцо является кольцом главных идеалов
\subsection{
    Теорема о том, что любое евклидово кольцо является кольцом главных идеалов.
}

% Лемма Гаусса
\subsection{
    Теорема Фробениуса (коммутативный случай).
}


    \section{
    Неприводимые многочлены. Расширение полей. Алгебраически замкнутые поля.
}

% Неприводимые многочлены
\subsection{
    Алгебры. Определения и примеры.
}

% Расширение полей
\subsection{
    Теорема Фробениуса (коммутативный случай).
}

% Алгебраически замкнутые поля
\subsection{
    Теорема Фробениуса (б/д).
}


    \section{
    Основная теорема алгебры (алгебраическая замкнутость поля комплексных чисел).
 }

% Основная теорема алгебры (алгебраическая замкнутость поля комплексных чисел)
\subsection{
    Алгебры. Определения и примеры.
}


    \section{
    Модули. Определение и примеры. Основные свойства. Векторное пространство, как модуль. 
 }

% Модули. Определение и примеры. Основные свойства
\subsection{
    Модули. Определение и примеры. Основные свойства.
}

% Векторное пространство, как модуль
\subsection{
    Векторное пространство, как модуль.
}


    \section{
    Теоремы о гомоморфизме для модулей. Аннулятор. 
}

% Теоремы о гомоморфизме для модулей
\subsection{
    Теоремы о гомоморфизме для модулей.
}

% Аннулятор
\subsection{
    Теорема Фробениуса (коммутативный случай).
}


    \section{
    Алгебры. Определения и примеры. Аналог теоремы Кели для алгебр.
}

% Алгебры. Определения и примеры
\subsection{
    Алгебры. Определения и примеры.
}

% Аналог теоремы Кели для алгебр
\subsection{
    Аналог теоремы Кели для алгебр.
}

    \section{
    Конечномерные алгебры. Минимальный многочлен элемента. Алгебры с делением. Обратимость элемента, не являющегося делителем нуля.
}

% Конечномерные алгебры
\subsection{
    Алгебры. Определения и примеры.
}

% Минимальный многочлен элемента
\subsection{
    Теорема Фробениуса (коммутативный случай).
}

% Алгебры с делением
\subsection{
    Теорема Фробениуса (б/д).
}

% Обратимость элемента, не являющегося делителем нуля
\subsection{
     Обратимость элемента, не являющегося делителем нуля.
}


    \section{
    Задание алгебры. Тело кватернионов. Теорема Фробениуса (б/д).
}

% Задание алгебры
\subsection{
    Алгебры. Определения и примеры.
}

% Тело кватернионов
\subsection{
    Теорема Фробениуса (коммутативный случай).
}

% Теорема Фробениуса (б/д)
\subsection{
    Теорема Фробениуса (б/д).
}


    \section{
    Алгебры с делением над полем комплексных чисел. Теорема Фробениуса (коммутативный случай). Групповая алгебра. Дифференцирование алгебр.
}

% Алгебры с делением над полем комплексных чисел
\subsection{
    Алгебры с делением над полем комплексных чисел
}

% Теорема Фробениуса (коммутативный случай)
\subsection{
    Теорема Фробениуса (коммутативный случай).
}

% Групповая алгебра
\subsection{
    Теорема Фробениуса (б/д).
}

% Дифференцирование алгебр
\subsection{
    Первая теорема о гомоморфизме.
}



\end{document}
